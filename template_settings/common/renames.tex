%%% Внесите свои данные - Input your data
%%
%%
\newcommand{\Author}{Д.В.\,Шергалис} % И.О. Фамилия автора
\newcommand{\AuthorFull}{Шергалис Донат Витальевич} % Фамилия Имя Отчество автора
\newcommand{\AuthorFullDat}{Шергалису Донату Вительевичу} % Фамилия Имя Отчество автора в дательном падеже (Кому? Студенту...)
\newcommand{\AuthorFullVin}{Шергалиса Доната Витальевича} % в винительном падеже (Кого? что?  Програмиста ...)
\newcommand{\AuthorPhone}{+7-921-758-45-08} % номер телефорна автора для оперативной связи
\newcommand{\Supervisor}{Г.В.\,Коваленко} % И. О. Фамилия научного руководителя
\newcommand{\SupervisorFull}{Коваленко Геннадий Васильевич} % Фамилия Имя Отчество научного руководителя
\newcommand{\SupervisorVin}{Г.В.\,Коваленко} % И. О. Фамилия научного руководителя  в винительном падеже (Кого? что? Руководителя ...)
\newcommand{\SupervisorJob}{доцент} %
\newcommand{\SupervisorJobVin}{доцента} % в винительном падеже (Кого? что?  Програмиста ...)
\newcommand{\SupervisorDegree}{к.э.н} %
\newcommand{\SupervisorTitle}{} %
%%
%%
%Руководитель, утверждающий задание
\newcommand{\Head}{А.В.\,Щукин} % И. О. Фамилия руководителя подразделения (руководителя ОП)
\newcommand{\HeadDegree}{Зам. директора}% Только должность:
%Руководитель %ОП 
%Заведующий % кафедрой
%Директор % Высшей школы
%Зам. директора
\newcommand{\HeadDep}{ВШИСиСТ} % заменить на краткую аббревиатуру подразделения или оставить пустым, если утверждает руководитель ОП

%%% Руководитель, принимающий заявление
\newcommand{\HeadAp}{А.В.\,Щукин} % И. О. Фамилия руководителя подразделения (руководителя ОП)
\newcommand{\HeadApDegree}{Зам. директора}% Только должность:
%Руководитель ОП 
%Заведующий кафедрой
%Директор Высшей школы
\newcommand{\HeadApDep}{ВШИСиСТ} % заменить на краткую аббревиатуру подразделения или оставить пустым, если утверждает руководитель ОП
%%% Консультант по нормоконтролю
\newcommand{\ConsultantNorm}{В.А.\,Пархоменко} % И. О. Фамилия консультанта по нормоконтролю. ТОЛЬКО из числа ППС!
\newcommand{\ConsultantNormDegree}{Ассистент ВШИСиСТ} %
%%% Первый консультант
\newcommand{\ConsultantExtraFull}{Кожевников Вадим Андреевич} % Фамилия Имя Отчетство дополнительного консультанта
\newcommand{\ConsultantExtra}{В.А.\,Кожевников} % И. О. Фамилия дополнительного консультанта
\newcommand{\ConsultantExtraDegree}{старший преподаватель} %
\newcommand{\ConsultantExtraVin}{А.А.\,Кожевникова} % И. О. Фамилия дополнительного консультанта в винительном падеже (Кого? что? Руководителя ...)
\newcommand{\ConsultantExtraDegreeVin}{старшего преподавателя} %  в винительном падеже (Кого? что? Руководителя ...)
%%% Второй консультант
\newcommand{\ConsultantExtraTwoFull}{Фамилия Имя Отчетство} % Фамилия Имя Отчетство дополнительного консультанта 
\newcommand{\ConsultantExtraTwo}{И.О.\,Фамилия} % И. О. Фамилия дополнительного консультанта 
\newcommand{\ConsultantExtraTwoDegree}{должность, степень} % 
\newcommand{\ConsultantExtraTwoVin}{И.О.\,Фамилию} % И. О. Фамилия дополнительного консультанта в винительном падеже (Кого? что? Руководителя ...)
\newcommand{\ConsultantExtraTwoDegreeVin}{должность, степень} %  в винительном падеже (Кого? что? Руководителя ...)
\newcommand{\Reviewer}{И.О.\,Фамилия} % И. О. Фамилия резензента. Обязателен только для магистров.
\newcommand{\ReviewerDegree}{должность, степень} % 
%%
%%
\renewcommand{\thesisTitle}{Разработка сервиса-адаптера между REST-клиентом и GraphQL-сервером}
\newcommand{\thesisDegree}{работа бакалавра}% дипломный проект, дипломная работа, магистерская диссертация %c 2020
\newcommand{\thesisTitleEn}{Development of an adaptor between REST clients and GraphQL servers} %2020
\newcommand{\thesisDeadline}{19.05.2021}
\newcommand{\thesisStartDate}{26.01.2021}
\newcommand{\thesisYear}{2021}
%%
%%
\newcommand{\group}{3530903/70301} % заменить вместо N номер группы
\newcommand{\thesisSpecialtyCode}{09.03.03}% код направления подготовки
\newcommand{\thesisSpecialtyTitle}{Прикладная информатика} % наименование направления/специальности
\newcommand{\thesisOPPostfix}{03} % последние цифры кода образовательной программы (после <<_>>)
\newcommand{\thesisOPTitle}{Прикладная информатика в области информационных ресурсов}% наименование образовательной программы
%%
%%
\newcommand{\institute}{
Институт компьютерных наук и~технологий
%Гуманитарный институт
%Инженерно-строительный институт
%Институт биомедицинских систем и технологий
%Институт металлургии, машиностроения и транспорта
%Институт передовых производственных технологий
%Институт прикладной математики и механики
%Институт физики, нанотехнологий и телекоммуникаций
%Институт физической культуры, спорта и туризма
%Институт энергетики и транспортных систем
%Институт промышленного менеджмента, экономики и торговли
}%
%%
%%




%%% Задание ключевых слов и аннотации
%%
%%
%% Ключевых слов от 3 до 5 слов или словосочетаний в именительном падеже именительном падеже множественного числа (или в единственном числе, если нет другой формы) по правилам русского языка!!!
%%
%%
\newcommand{\keywordsRu}{Архитектура клиент-серверного взаимодействия, GraphQL, REST, архитектура информационной системы, миграция на GraphQL}
%%
%%
\newcommand{\keywordsEn}{Architecture for client-server interaction, GraphQL, REST, information system architecture, migration to GraphQL} % ВВЕДИТЕ ключевые слова по-английски
%%
%%
%% Реферат ОТ 1000 ДО 1500 знаков на русский или английский текст
%%
%Реферат должен содержать:
%- предмет, тему, цель ВКР;
%- метод или методологию проведения ВКР:
%- результаты ВКР:
%- область применения результатов ВКР;
%- выводы.

\newcommand{\abstractRu}{В данной работе рассматривается новый подход к организации клиент-серверного взаимодействия.
В первой главе произведён обзор самых популярных современных технологий для организации клиент-серверного взаимодействия -- REST и GraphQ, приведено их описание, перечислены достоинства и недостатки каждой технологии, а также произведено сравнение этих технологий между собой, и описаны возможные сложности, которые могут возникнуть при миграции с REST на GraphQL.
Во второй главе представлена идея сервиса-адаптера, который не имеет многих недостатков перечисленных технологий, но при этом сохраняет их достоинства. По этой идее сформулирована спецификация, и на основании неё приведено перечисление ожидаемых преимуществ и недостатков. На основании них принято решение о целесообразности реализации описанного сервиса, и в третьей главе приводится описание процесса разработки сервиса, соответствующего описанной спецификации. В четвёртой главе приводится описание процесса и результатов тестирования разработанного сервиса, а также рассматриваются варианты его дальнейшего улучшения. } % ВВЕДИТЕ текст аннотации по-русски
%%
%%
\newcommand{\abstractEn}{This paper discusses a new approach to organizing client-server interaction.
The first chapter provides an overview of the most popular modern technologies for organizing client-server interaction - REST and GraphQ, provides their description, lists the advantages and disadvantages of each technology, and also compares these technologies with each other, and describes the possible difficulties that may arise when migrations from REST to GraphQL.
In the second chapter, the idea of a service adapter is presented, which does not have many of the disadvantages of the listed technologies, but at the same time retains their advantages. Based on this idea, a specification is formulated, and on the basis of it, a listing of the expected advantages and disadvantages is given. Based on them, a decision was made about the feasibility of implementing the described service, and the third chapter describes the process of developing a service that meets the described specification. The fourth chapter describes the process and results of testing the developed service, and also discusses options for its further improvement.} % TODO fix english



%%% РАЗДЕЛ ДЛЯ ОФОРМЛЕНИЯ ПРАКТИКИ
%Место прохождения практики
\newcommand{\PracticeType}{Отчет о прохождении % 
	%стационарной производственной (технологической (проектно-технологической)) %
	преддипломной % тип и вид ЗАМЕНИТЬ
	практики}

\newcommand{\Workplace}{СПбПУ, ИКНТ, ВШИСиСТ} % TODO Rename this variable

% Даты начала/окончания
\newcommand{\PracticeStartDate}{%
06.05.2021%
%	22.06.2020
}%
\newcommand{\PracticeEndDate}{%
	20.05.2021%
%	18.07.2020%
}%
%%

\newcommand{\School}{
	Высшая школа интеллектуальных систем и~суперкомпьютерных~технологий
}
\newcommand{\practiceTitle}{Разработка сервиса-адаптера между REST-клиентом и GraphQL-сервером}


%% ВНИМАНИЕ! Необходимо либо заменить текст аннотации (ключевых слов) на русском и английском, либо удалить там весь текст, иначе в свойства pdf-отчета по практике пойдет шаблонный текст.

%%% Не меняем дальнейшую часть - Do not modify the rest part
%%
%%
%%
%%
\ifnumequal{\value{docType}}{1}{% Если ВКР, то...
	\newcommand{\DocType}{Выпускная квалификационная работа}
	\newcommand{\pdfDocType}{\DocType~(\thesisDegree)} %задаём метаданные pdf файла
	\newcommand{\pdfTitle}{\thesisTitle}
}{% Иначе 
	\newcommand{\DocType}{\PracticeType}
	\newcommand{\pdfDocType}{\DocType} %задаём метаданные pdf файла
	\newcommand{\pdfTitle}{\practiceTitle}
}%
\newcommand{\HeadTitle}{\HeadDegree~\HeadDep}
\newcommand{\HeadApTitle}{\HeadApDegree~\HeadApDep}
\newcommand{\thesisOPCode}{\thesisSpecialtyCode\_\thesisOPPostfix}% код образовательной программы
\newcommand{\thesisSpecialtyCodeAndTitle}{\thesisSpecialtyCode~\thesisSpecialtyTitle}% Код и наименование направления/специальности
\newcommand{\thesisOPCodeAndTitle}{\thesisOPCode~\thesisOPTitle} % код и наименование образовательной программы
%%
%%
\hypersetup{%часть болка hypesetup в style
		pdftitle={\pdfTitle},    % Заголовок pdf-файла
		pdfauthor={\AuthorFull},    % Автор
		pdfsubject={\pdfDocType. Шифр и наименование направления подготовки: \thesisSpecialtyCodeAndTitle. \abstractRu},      % Тема
		pdfcreator={LaTeX, SPbPU-student-thesis-template},     % Приложение-создатель
%		pdfproducer={},  % Производитель, Производитель PDF % будет выставлена автоматически
		pdfkeywords={\keywordsRu}
}
%%
%%
%% вспомогательные команды
\newcommand{\firef}[1]{рис.\ref{#1}} %figure reference
\newcommand{\taref}[1]{табл.\ref{#1}}	%table reference
%%
%%
%% Архивный вариант задания ключевых слов, аннотации и благодарностей 
% Too hard to export data from the environment to pdf-info
% https://tex.stackexchange.com/questions/184503/collecting-contents-of-environment-and-store-them-for-later-retrieval
%заменить NewEnviron на newenvironment для распознавания команды в TexStudio
%\NewEnviron{keywordsRu}{\noindent\MakeUppercase{\BODY}}
%\NewEnviron{keywordsEn}{\noindent\MakeUppercase{\BODY}}
%\newenvironment{abstractRu}{}{}
%\newenvironment{abstractEn}{}{}
%\newenvironment{acknowledgementsRu}{\par{\normalfont \acknowledgements.}}{}
%\newenvironment{acknowledgementsEn}{\par{\normalfont \acknowledgementsENG.}}{}


%%% Переопределение именований %%% Не меняем - Do not modify
%\newcommand{\Ministry}{Минобрнауки России} 
\newcommand{\Ministry}{Министерство науки и высшего образования Российской~Федерации} %с 2020
\newcommand{\SPbPU}{Санкт-Петербургский политехнический университет Петра~Великого}
\newcommand{\SPbPUOfficialPrefix}{Федеральное государственное автономное образовательное учреждение высшего образования}
\newcommand{\SPbPUOfficialShort}{ФГАОУ~ВО~<<СПбПУ>>}
%% Пробел между И. О. не допускается.
\renewcommand{\alsoname}{см. также}
\renewcommand{\seename}{см.}
\renewcommand{\headtoname}{вх.}
\renewcommand{\ccname}{исх.}
\renewcommand{\enclname}{вкл.}
\renewcommand{\pagename}{Pages}
\renewcommand{\partname}{Часть}
\renewcommand{\abstractname}{\textbf{Аннотация}}
\newcommand{\abstractnameENG}{\textbf{Annotation}}
\newcommand{\keywords}{\textbf{Ключевые слова}}
\newcommand{\keywordsENG}{\textbf{Keywords}}
\newcommand{\acknowledgements}{\textbf{Благодарности}}
\newcommand{\acknowledgementsENG}{\textbf{Acknowledgements}}
\renewcommand{\contentsname}{Content} % 
%\renewcommand{\contentsname}{Содержание} % (ГОСТ Р 7.0.11-2011, 4)
%\renewcommand{\contentsname}{Оглавление} % (ГОСТ Р 7.0.11-2011, 4)
\renewcommand{\figurename}{Рис.} % Стиль СПбПУ
%\renewcommand{\figurename}{Рисунок} % (ГОСТ Р 7.0.11-2011, 5.3.9)
\renewcommand{\tablename}{Таблица} % (ГОСТ Р 7.0.11-2011, 5.3.10)
%\renewcommand{\indexname}{Предметный указатель}
\renewcommand{\listfigurename}{Список рисунков}
\renewcommand{\listtablename}{Список таблиц}
\renewcommand{\refname}{\fullbibtitle}
\renewcommand{\bibname}{\fullbibtitle}

\newcommand{\chapterEnTitle}{Сhapter title} % <- input the English title here (only once!) 
\newcommand{\chapterRuTitle}{Название главы}          % <- введите 
\newcommand{\sectionEnTitle}{Section title} %<- input subparagraph title in english
\newcommand{\sectionRuTitle}{Название подраздела} % <- введите название подраздела по-русски
\newcommand{\subsectionEnTitle}{Subsection title} % - input subsection title in english
\newcommand{\subsectionRuTitle}{Название параграфа} % <- введите название параграфа по-русски
\newcommand{\subsubsectionEnTitle}{Subsubsection title} % <- input subparagraph title in english
\newcommand{\subsubsectionRuTitle}{Название подпараграфа} % <- введите название подпараграфа по-русски