%! suppress = Unicode
\chapter*{Введение} % * не проставляет номер
\addcontentsline{toc}{chapter}{Введение} % вносим в содержание

В течение многих лет основными архитектурными подходами, использующимися в сфере информационных технологий для обмена данными между системами, являлись такие технологии как SOAP (Simple Object Access Protocol) и REST (Representational state transfer).

Очевидным их отличием является используемый способ представления данных.
В случае SOAP единственным поддерживаемым форматом является XML, а в случае REST могут использоваться форматы JSON, HTML и XML. Каждый из подходов имеет свои достоинства и недостатки, однако на данный момент REST является приоритетным выбором для современных разработчиков в силу легковесности и лучшей читаемости формата JSON, гибкости и простоте использования, особенно в Web-разработке, где он обязан своей популярности языку JavaScript.

Однако при работе с REST разработчики также столкнулись с определёнными ограничениями и недостатками.
Основными из них считаются так называемые overfetching и underfetching, когда в ответ на запрос REST-клиент получает лишние данные, или наоборот недостаточное их количество, из-за чего клиент вынужден выполнять другой запрос.
Это часто случается, когда меняется дизайн приложения, или изменяется или добавляется новая функциональность, переиспользующая ранее существовавшие запросы.
В случае недостаточности данных разработчику серверной части приходится вносить изменения в код, чтобы добавить требуемые данные, а избыточность зачастую игнорируется.
Проблема усугубляется наличием нескольких типов клиентов – например, Android, iOS и веб-приложений.

Необходимость постоянных небольших изменений на серверной стороне приводит к замедлению процесса разработки.
Как один из способов решения этой проблемы в компании Facebook для внутреннего использования был создан язык запросов GraphQL, позволивший разработчикам на стороне клиента самостоятельно выбирать те данные, которые им необходимо получить, с помощью специального языка запросов.

После публикации спецификации в открытый доступ, разработчики заинтересовались новым гибким подходом, однако столкнулись с рядом препятствий при попытке внедрить данную технологию в разрабатываемые ими системы.
Основными из них стали:
\begin{itemize}
    \item невозможность переработки прошлых версий мобильных приложений для использования GraphQL;
	\item необходимость в поддержке обоих протоколов на время миграции;
	\item неготовность разработчиков клиентских приложений изучать и внедрять новую технологию.
\end{itemize}

Для решения указанных проблем автор предлагает создать сервис, который станет посредником между GraphQL-сервером и REST-клиентом.
Автор ожидает, что в этом случае будут решены перечисленные проблемы, а также получены дополнительные преимущества по сравнению с использованием каждой технологии в отдельности.

\section{Постановка задачи}\label{intro:tasks}

Целью данной работы является создание прототипа указанного сервиса-прослойки.
Для этого автор должен решить следующие задачи:
\begin{itemize}
	\item Изучить технологии REST и GraphQL;
	\item Осуществить сравнение GraphQL с другими архитектурами и протоколами;
	\item Составить техническое задание для реализации сервиса;
	\item Подготовить данные и окружение для тестирования;
	\item Разработать прототип сервиса и протестировать его на подготовленных данных.
\end{itemize}

%% Вспомогательные команды - Additional commands
%\newpage % принудительное начало с новой страницы, использовать только в конце раздела
%\clearpage % осуществляется пакетом <<placeins>> в пределах секций
%\newpage\leavevmode\thispagestyle{empty}\newpage % 100 % начало новой строки