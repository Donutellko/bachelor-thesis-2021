%! suppress = Unicode
\chapter*{Словарь терминов}             % Заголовок
\addcontentsline{toc}{chapter}{Словарь терминов}  % Добавляем его в оглавление

\textbf{REST} --- архитектурный стиль построения взаимодействия между клиентом и сервисом.

\textbf{SOAP} --- протокол удалённого вызова процедур, передачи сообщений между системами в формате XML\@.

\textbf{GraphQL} --- язык запросов к серверу, использующий строгий синтаксис и схемы данных.

\textbf{Фронтенд (frontend)} --- клиентская часть системы, с которой напрямую осуществляет взаимодействие конечный пользователь.
Им может являться например мобильное, десктопное или веб-приложение.

\textbf{Бэкенд (backend)} --- серверная часть системы, обычно представляет собой набор сервисов.

\textbf{Сервис} --- серверное приложение, обрабатывающее запросы клиента.

\textbf{Эндпоинт (endpoint)} --- одна из, или единственная точка входа для запросов к сервису.

\textbf{Маппинг (mapping)} --- описание соответствия между двумя сущностями.

\textbf{Бин (bean)} --- объект, который управляется по принципу инверсии контроля (Inversion of Control) неким IoC-контейнером.
В данной работе роль IoC-контейнера выполняет Spring Framework.
