%! suppress = Unicode
\chapter{ТЕСТИРОВАНИЕ РЕАЛИЗАЦИИ} \label{ch4}

\textbf{Хорошим стилем является наличие введения к главе, которое \textit{начинается непосредственно после названия главы, без оформления в виде отдельного параграфа}.}

\section{Название параграфа} \label{ch4:sec1}

Для проверки работоспособности прототипа были использованы тестовые данные, подготовленные в разделе 2.3. Проверяются следующие сценарии:

\begin{itemize}
	\item Для запроса существует маппинг, не содержащий переменных.
	Система отправляет GraphQL-сервису шаблон из маппинга в неизменном виде и возвращает пользователю ответ.

	\item Для запроса существует маппинг, содержащий несколько переменных.
	Система заполняет шаблон этими переменными и отправляет результат GraphQL-сервису, возвращает пользователю ответ.

    \item Для запроса не существует маппинг.
	Запрос в неизменном виде перенаправляется по адресу Gateway API, указанному в настройках.
	В данном случае использовался сервис Postman Echo, возвращающий body и headers запроса в качестве ответа.
	Ответ этого сервиса возвращается пользователю.
\end{itemize}

Все сценарии были успешно проверены, в результате чего прототип считается работоспособным и будет использован для реализации целевой системы.


\section{Выводы} \label{ch4:conclusion}

Текст выводов по главе \thechapter.
