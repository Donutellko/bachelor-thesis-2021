%! suppress = Unicode
\chapter{ТЕСТИРОВАНИЕ РЕАЛИЗАЦИИ} \label{ch4}

\textbf{Хорошим стилем является наличие введения к главе, которое \textit{начинается непосредственно после названия главы, без оформления в виде отдельного параграфа}.}

\section{Данные для тестирования реализации}\label{sec:testing-data}

Для тестирования сервиса R2G планируется использовать публичные GraphQL-сервисы.
В частности, сервис FakeQL позволяет создать GraphQL-сервис на основании данных в виде JSON. Для этого используем JSON, представленный в приложении 5.
Данный JSON загружается на сайт FakeQL, после чего мы получаем URL эндпоинта, к которому можно отправлять GraphQL-запросы.
Схема получившегося сервиса была продемонстрирована ранее (см. приложение 1).

В приложении 6 продемонстрирован пример маппинга.
для получения списка счетов в валюте из query parameter, для пользователя с идентификатором, переданным в URI. В приложении 7 представлен пример запроса, который соответствует указанному маппингу, а также GraphQL-запрос, который должен быть отправлен в GraphQL-сервер при получении такого запроса.

\section{Название параграфа} \label{ch4:sec1}

Для проверки работоспособности прототипа были использованы тестовые данные, подготовленные в разделе 2.3. Проверяются следующие сценарии:

\begin{itemize}
	\item Для запроса существует маппинг, не содержащий переменных.
	Система отправляет GraphQL-сервису шаблон из маппинга в неизменном виде и возвращает пользователю ответ.

	\item Для запроса существует маппинг, содержащий несколько переменных.
	Система заполняет шаблон этими переменными и отправляет результат GraphQL-сервису, возвращает пользователю ответ.

    \item Для запроса не существует маппинг.
	Запрос в неизменном виде перенаправляется по адресу Gateway API, указанному в настройках.
	В данном случае использовался сервис Postman Echo, возвращающий body и headers запроса в качестве ответа.
	Ответ этого сервиса возвращается пользователю.
\end{itemize}

Все сценарии были успешно проверены, в результате чего прототип считается работоспособным и будет использован для реализации целевой системы.


\section{Выводы} \label{ch4:conclusion}

Текст выводов по главе \thechapter.
