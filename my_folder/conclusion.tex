%! suppress = Unicode
\chapter*{Заключение} \label{ch-conclusion}
\addcontentsline{toc}{chapter}{Заключение}	% в оглавление 

В процессе выполнения научно-исследовательской работы были изучены принципы технологий REST и GraphQL, определены их преимущества и недостатки, произведено сравнение между собой.
Смоделирован процесс миграции между данными технологиями и определены проблемы, возникающие при этом.

На основании полученных данных сформулированы принципы, лежащие в основе системы, которая должна стать альтернативой использованию каждой из указанных технологий в отдельности, а также упростить осуществлении миграции.

Для описанной системы определены её предполагаемые преимущества и недостатки, на основании чего принято решение о целесообразности реализации данной системы в процессе выполнения выпускной квалификационной работы.
Для этого было сформулировано техническое задание, включающее в себя требования к готовой системе, а также подготовлен пример тестовых данных, которые могут быть использованы для проверки работоспособности готовой системы.

По основным требованиям, сформулированным в техническом задании, был реализован и проверен на тестовых данных прототип веб-сервиса, который ляжет в основу готовой системы.
