%! suppress = Unicode


\chapter{ПРЕДЛАГАЕМАЯ ТЕХНОЛОГИЯ} \label{ch:ch2}

В этой главе мы представим технологию, предназначенную для решения проблем, описанных в предыдущей главе.
Рассмотрим её предполагаемые преимущества и недостатки, и примем решение о целесообразности разработки такой системы.
Затем мы сформулируем спецификацию для возможных реализаций данной технологии.


\section{Предлагаемая технология}\label{sec:proposed-technology}

В качестве решения для перечисленных проблем автор предлагает создание дополнительного сервиса в виде адаптера между REST-клиентом и GraphQL-сервером.
Данный сервис должен хранить маппинг (соответствие между запросами двух типов), и при получении REST-запроса находить в своей базе соответствующий шаблон GraphQL-запроса, заполнять его данными из REST-запроса и отправлять к GraphQL-сервису для выполнения.

Помимо описанной основной функциональности также возможна реализация следующих функциональностей:

\begin{itemize}
    \item В случае, если для REST-запроса не было найдено соответствия, то запрос должен быть передан gateway-сервису – таким образом, описываемый сервис может стать единственной точкой доступа к системе.
    \item При необходимости может быть реализовано приведение ответа GraphQL-сервиса к другому виду с целью сохранения обратной совместимости.
\end{itemize}

\subsection{Достоинства предлагаемой технологии}\label{subsec:proposed-technology-advantages}

Рассмотрим достоинства данного подхода в сравнении с использованием REST либо GraphQL:

\begin{itemize}
    \item Гибкость и мощь языка GraphQL – разработчики имеют возможность писать полноценные GraphQL-запросы и разрабатывать серверную часть в соответствии со всеми принципами GraphQL\@.
    \item Обратная совместимость на клиенте – старая версия приложения может продолжать функционировать после полного перехода на использование GraphQL на сервере.
    \item Разработчикам клиентских приложений не нужно обучаться новой технологии и внедрять её – взаимодействие с сервером по-прежнему осуществляется через привычный REST\@.
    \item Обязанности по написанию, отладке и редактированию запросов могут брать на себя также аналитики и сотрудники отдела сопровождения, не требуя участия разработчиков.
    \item Возможность постепенной миграции – запросы, которые не были реализованы в GraphQL, передаются старым сервисам, и затем неявно подменяются новой реализацией.
    \item Такой сервис может обращаться ко множеству различных GraphQL-сервисов, позволяя разграничить зоны ответственности и решить проблему именования \cite{graphql-disadvantage-namespaces}.
    \item Есть возможность использовать существующие механизмы кеширования на клиенте и промежуточных узлах, непригодные для использования с GraphQL\@.
    \item Подобная система может быть реализована для любого протокола общения с клиентом – например, вместо REST может использоваться протокол SOAP\@.
    \item Для добавления и изменения маппинга необязательна перезагрузка сервиса, что позволяет вносить и проверять изменения гораздо быстрее, чем это происходит при изменении кода.
    Актуализация маппинга может происходить периодически, при обнаружении изменений или по иному событию.
    \item Возможность редактирования запроса на сервере в реальном времени, что невозможно при использовании механизма Persistent Queries, при котором клиент использует хэш-код одного из заранее заданных неизменяемых запросов.
    \item Настройки могут браться из любого типа источника – git или другой системы контроля версий, базы данных, config-серверов, локальных yaml-файлов и т.п.
    \item Хранение маппингов в системе контроля версий позволит использовать уже имеющуюся ролевую модель, в том числе налаженные процессы для ревью изменений, а также защитить маппинги от несанкционированного изменения и утраты, использовать версионирование.
    \item Форматом ответа можно легко манипулировать с помощью технологий для форматирования JSON – например, библиотеки JOLT\cite{jolt-github}.
    \item В случае миграции с данного решения на использование GraphQL без посредников разработчики будут иметь в своём распоряжении готовые и протестированные GraphQL-запросы.
    А для поддержки старых версий приложения после миграции не придётся поддерживать старый REST-кластер, а только данный сервис.
    \item Уменьшается объём кода, отвечающего за отображение, в сервисах.
    Появляется новый уровень абстракции, позволяющий разработчикам писать более чистый код.
    \item Данный сервис в качестве отдельного слоя может быть протестирован независимо от остальной системы при помощи автотестов: для этого могут использоваться «заглушки», вызываемые вместо реальных сервисов на тестовом стенде.
\end{itemize}

\subsection{Недостатки предлагаемой технологии}\label{subsec:proposed-technology-disadvantages}

\begin{itemize}
    \item Дополнительный слой в цепочке вызовов замедляет выполнение запросов (увеличивается latency).
    Фактическое время исполнения запросов предлагаемым сервисом будет измерено и оценено при его апробации.

    \item Сервис и источники маппингов являются новыми потенциальными точками отказа.
    При ошибке в маппинге или программном или аппаратном сбое вся система или её часть может оказаться недоступной.
    Для увеличения отказоустойчивости можно применить горизонтальное масштабирование, увеличивая число реплик, а также размещая их в различных датацентрах.
    Также можно разделять ответственность за разные части системы между разными кластерами предлагаемого сервиса, чтобы уменьшить возможные последствия.

    \item В случае недоступности источника маппингов сервис не сможет быть перезапущен.
    Например, это может произойти при отказе базы данных с маппингами или недоступности GIT-репозитория.
    Для защиты от данного риска следует предусмотреть резервный источник.
    Им может быть балансировщик со включенным кешированием или дополнительная база данных.

    \item При росте количества маппингов может значительно возрасти сложность поддержки системы.
    Для упрощения управления маппингами следует логически разделять

    \item Нужен новый механизм тестирования изменений.
    Маппинги создаются и изменяются в процессе работы сервиса.
    Поэтому правильность задания каждого конкретного маппинга может быть проверена только через запуск сервиса.
    Запустить сервис можно как на тестировочном стенде, так и при сборке приложения в рамках интеграционного тестирования.
    Однако первый вариант достаточно ресурсоёмкий, а второй требует навыков разработки.
    Поэтому для возможности добавления, изменения и проверки маппингов людьми без опыта разработки следует разработать специальное программное обеспечение, которое будет получать список возможных запросов, а также имеющихся маппингов, чтобы проверить их корректность, а также то, что для каждого запроса существует не более одного подходящего маппинга.

    \item Создание, изменения и проверки маппингов требует специальных навыков помимо знания языка GraphQL\@.
    Способ решения этого недостатка описан в предыдущем пункте.

\end{itemize}

Как видно, количество ожидаемых преимуществ значительно превышает количество ожидаемых недостатков, в связи с чем разработку подобной системы считаем целесообразной.


\section{Требования к реализации} \label{sec:requirements} %название по-русски

Реализация предлагаемой системы в рамках выпускной квалификационной работы должна обладать следующей функциональностью:

\begin{itemize}
    \item Загружать маппинги из источника, которым является git-репозиторий, ссылка на который указывается в настройках.

    \item Принимать REST-запросы, имеющие методы \texttt{GET}, \texttt{POST}, \texttt{PUT}, \texttt{DELETE}.

    \item Находить для запроса соответствующий маппинг по методу запроса и URI запроса.
    URI запроса, указанный в маппинге, может иметь \texttt{path variable}, то есть для следующего запроса

    \texttt{GET /api/users/123/accounts?currency=RUR}

    должен быть найден следующий маппинг:

    \texttt{GET /api/users/\#\{userId\}/accounts}

    \item Использовать \texttt{path variables}, \texttt{query params} и заголовки запроса в качестве переменных для формирования запроса по шаблону, указанному в маппинге.
    Так, в предыдущем примере на основании запроса будут задаваться значения двух переменных: \texttt{userId=123} и \texttt{currency=RUR}.

    \item Также использовать тело для получения значений переменных.
    Например, при получении запроса со следующим телом

    \texttt{\{ "name": "Александров"\, "address": \{"street": "Лубянка"\, "house": "1" \}\}}

    будут получены значения переменных

    \texttt{name=Александров ; address.street=Лубянка ; address.house=1}

    \item Подставлять значения переменных в шаблон запроса, хранимый в маппинге, на место плейсхолдеров с соответствующий именем.
    Например, следующий запрос из маппинга:

    \texttt{\{ users(id: "\#\{userId\}")\{ accounts(currency\_eq: "\#{currency}") \{ number \} \}}

    должен быть заполнен следующим образом:

    \texttt{\{ users(id: "123")\{ accounts(currency\_eq: "RUR") \{ number \} \}}

    \item В случае, если маппинг не был найден, запрос должен быть в неизменном виде (с сохранением URI, \texttt{query params}, заголовков и тела запроса) направлен по адресу API Gateway, заданному в настройках.

    \item При наличии нескольких подходящих маппингов, выбирать более специфичный.
    Например, при наличии маппингов

    \texttt{GET /users/\#\{userId\}} и \texttt{GET /users/\#\{userId\}/accounts/\#\{accountId\}}

    при поступлении запроса

    \texttt{GET /users/123/accounts/456}

    должен быть использован второй маппинг.

    \item Осуществить преобразование полученного ответа осуществляется при наличии правила для преобразования в маппинге.
    Для преобразования используется технология, указанная в маппинге.
    Правила для осуществления преобразования задаются в соответствующем указанной технологии формате.
    Выбор поддерживаемых технологий осуществляется разработчиком.
% TODO Дописать про использование GraphQLных variables
\end{itemize}

% TODO Диаграмма, иллюстрирующая взаимодействие между клиентом и описываемым сервисом изображена в приложении 4.

\section{Выводы} \label{sec:ch2-conclusion}

В рамках этой главы была представлена новая технология, перечислены и взвешены её предполагаемые достоинства и недостатки, и на основании их принято решение о том, что реализация описанной технологии является целесообразной и полезной.
Затем было более подробно описано предполагаемое поведение системы при получении различных запросов, и это описание может быть использовано в дальнейшем в качестве спецификации при реализации системы.
